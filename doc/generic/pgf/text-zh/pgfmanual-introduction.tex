% Copyright 2019 by Till Tantau
% Chinese translation copyright (c) 2021 by rnicrosoft-studio
%
% This file may be distributed and/or modified
%
% 1. under the LaTeX Project Public License and/or
% 2. under the GNU Free Documentation License.
%
% See the file doc/generic/pgf/licenses/LICENSE for more details.


\section{引言 Introduction}

欢迎阅读\tikzname 和底层\pgfname 系统的文档。
一开始只是在我(Till Tantau)的博士论文中使用pdf\LaTeX 创建图形的小小\LaTeX style,现在已经发展成为一种成熟的图形语言,拥有超过1000页的手册。
\tikzname 提供的丰富的选择常常让初学者望而却步;
但幸运的是,本文档附带了一些慢节奏的教程,它们将教你关于\tikzname 应该知道的几乎所有内容,而无需阅读剩余内容。

我想从问题“什么是\tikzname?”开始说起。
基本上,它只是定义了一些\TeX 绘制图形命令。
比方说,代码|\tikz \draw (0pt,0pt) -- (20pt,6pt);|产生线段\tikz \draw (0pt,0pt) -- (20pt,6pt);,代码|\tikz \fill[orange] (1ex,1ex) circle (1ex);|产生\tikz \fill[orange] (1ex,1ex) circle (1ex);。
在某种意义上,当你使用\tikzname 时你是在“编程”你的图形,就如同当你使用\TeX 时你是在“编程”你的文档一样。
这也解释了名字:\tikzname 是一个承袭“\textsc{gnu}不是Unix”传统的递归首字母缩写,意思是“\tikzname\ ist \emph{kein} Zeichenprogramm”\footnote{译注:德语},翻译过来就是“\tikzname 不是一个绘图程序”,告诫读者该期待什么。
使用\tikzname,你可以为你的图形获得“\TeX 排版方法”的所有优点:快速创建简单的图形,精确的定位,宏的使用,往往优越的排版。
你继承了所有的缺点:陡峭的学习曲线,没有\textsc{wysiwyg}(所见即所得),小的更改需要很长时间重新编译,以及代码不能真正“展示”事情的样子。

现在我们知道了什么是\tikzname,那么“\pgfname”呢?
如前所述,\tikzname最初是一个实现\TeX图形宏的项目,既可以用于pdf\LaTeX,也可以用于经典的(基于PostScript)\LaTeX。
换句话说,我希望为\TeX实现一种“portable graphics format(可移植的图形格式)”——因此命名为\pgfname。
这些早期的宏仍然存在,它们构成了本手册中描述的系统的“基础层”,但一个作者目前大多数的交互是与\tikz进行的——它自己已经成为一门独立的语言。


\subsection{\tikzname之下的层}

在\tikzname之下实际上有\emph{两个}层:
%
\begin{description}
    \item[System layer(系统层):] This layer provides a complete abstraction of what
        is going on ``in the driver''. The driver is a program like |dvips|
        or |dvipdfm| that takes a |.dvi| file as input and generates a |.ps|
        or a |.pdf| file. (The |pdftex| program also counts as a driver, even
        though it does not take a |.dvi| file as input. Never mind.) Each
        driver has its own syntax for the generation of graphics, causing
        headaches to everyone who wants to create graphics in a portable way.
        \pgfname's system layer ``abstracts away'' these differences. For
        example, the system command |\pgfsys@lineto{10pt}{10pt}| extends the
        current path  to the coordinate $(10\mathrm{pt},10\mathrm{pt})$ of
        the current |{pgfpicture}|. Depending on whether |dvips|, |dvipdfm|,
        or |pdftex| is used to process the document, the system command will
        be converted to different |\special| commands. The system layer is as
        ``minimalistic'' as possible since each additional command makes it
        more work to port \pgfname\ to a new driver.

        As a user, you will not use the system layer directly.
    \item[Basic layer(基础层):] The basic layer provides a set of basic commands that
        allow you to produce complex graphics in a much easier manner than by
        using the system layer directly. For example, the system layer provides
        no commands for creating circles since circles can be composed from the
        more basic Bézier curves (well, almost). However, as a user you will
        want to have a simple command to create circles (at least I do) instead
        of having to write down half a page of Bézier curve support
        coordinates. Thus, the basic layer provides a command |\pgfpathcircle|
        that generates the necessary curve coordinates for you.

        The basic layer consists of a \emph{core}, which consists of several
        interdependent packages that can only be loaded \emph{en bloc}, and
        additional \emph{modules} that extend the core by more
        special-purpose commands like node management or a plotting
        interface. For instance, the \textsc{beamer} package uses only the
        core and not, say, the |shapes| modules.
\end{description}

In theory, \tikzname\ itself is just one of several possible ``frontends''.
which are sets of commands or a special syntax that makes using the basic layer
easier. A problem with directly using the basic layer is that code written for
this layer is often too ``verbose''. For example, to draw a simple triangle,
you may need as many as five commands when using the basic layer: One for
beginning a path at the first corner of the triangle, one for extending the
path to the second corner, one for going to the third, one for closing the
path, and one for actually painting the triangle (as opposed to filling it).
With the \tikzname\ frontend all this boils down to a single simple
\textsc{metafont}-like command:
%
\begin{verbatim}
\draw (0,0) -- (1,0) -- (1,1) -- cycle;
\end{verbatim}

In practice, \tikzname\ is the only ``serious'' frontend for \pgfname. It gives
you access to all features of \pgfname, but it is intended to be easy to use.
The syntax is a mixture of \textsc{metafont} and \textsc{pstricks} and some
ideas of myself. There are other frontends besides \tikzname, but they are intended
more as ``technology studies'' and less as serious alternatives to
\tikzname. In particular, the |pgfpict2e| frontend   reimplements the standard
\LaTeX\ |{picture}|  environment and commands like |\line| or |\vector| using
the \pgfname\ basic layer. This layer is not really ``necessary'' since the
|pict2e.sty| package does at least as good a job at reimplementing the
|{picture}| environment. Rather, the idea behind this package is to have a
simple demonstration of how a frontend can be implemented.

Since most users will only use \tikzname\ and almost no one will use the system
layer directly, this manual is mainly about \tikzname\ in the first parts; the
basic layer and the system layer are explained at the end.


\subsection{与其他图形包的比较}

\tikzname\ is not the only graphics package for \TeX. In the following, I try
to give a reasonably fair comparison of \tikzname\ and other packages.
%
\begin{enumerate}
    \item The standard \LaTeX\ |{picture}| environment allows you to create
        simple graphics, but little more. This is certainly not due to a lack
        of knowledge or imagination on the part of \LaTeX's designer(s).
        Rather, this is the price paid for the |{picture}| environment's
        portability: It works together with all backend drivers.
    \item The |pstricks| package is certainly powerful enough to create any
        conceivable kind of graphic, but it is not really portable. Most
        importantly, it does not work with |pdftex| nor with any other driver
        that produces anything but PostScript code.

        Compared to \tikzname, |pstricks| has a similar support base. There
        are many nice extra packages for special purpose situations that have
        been contributed by users over the last decade. The \tikzname\ syntax
        is more consistent than the |pstricks| syntax as \tikzname\ was
        developed ``in a more centralized manner'' and also ``with the
        shortcomings on |pstricks| in mind''.
    \item The |xypic| package is an older package for creating graphics.
        However, it is more difficult to use and to learn because the syntax
        and the documentation are a bit cryptic.
    \item The |dratex| package is a small graphic package for creating a
        graphics. Compared to the other package, including \tikzname, it is
        very small, which may or may not be an advantage.
    \item The |metapost| program is a powerful alternative to \tikzname. It
        used to be an external program, which entailed a bunch of problems,
        but in Lua\TeX\ it is now built in. An obstacle with |metapost| is
        the inclusion of labels. This is \emph{much} easier to achieve using
        \pgfname.
    \item The |xfig| program is an important alternative to \tikzname\ for
        users who do not wish to ``program'' their graphics as is necessary
        with \tikzname\ and the other packages above. There is a conversion
        program that will convert |xfig| graphics to \tikzname.
\end{enumerate}


\subsection{工具包}

The \pgfname\ package comes along with a number of utility package that are not
really about creating graphics and which can be used independently of \pgfname.
However, they are bundled with \pgfname, partly out of convenience, partly
because their functionality is closely intertwined with \pgfname. These utility
packages are:
%
\begin{enumerate}
    \item The |pgfkeys| package defines a powerful key management facility.
        It can be used completely independently of \pgfname.
    \item The |pgffor| package defines a useful |\foreach| statement.
    \item The |pgfcalendar| package defines macros for creating calendars.
        Typically, these calendars will be rendered using \pgfname's graphic
        engine, but you can use |pgfcalendar| also typeset calendars using
        normal text. The package also defines commands for ``working'' with
        dates.
    \item The |pgfpages| package is used to assemble several pages into a
        single page. It provides commands for assembling several ``virtual
        pages'' into a single ``physical page''. The idea is that whenever
        \TeX\ has a page ready for ``shipout'', |pgfpages| interrupts this
        shipout and instead stores the page to be shipped out in a special
        box. When enough ``virtual pages'' have been accumulated in this way,
        they are scaled down and arranged on a ``physical page'', which then
        \emph{really} shipped out. This mechanism allows you to create ``two
        page on one page'' versions of a document directly inside \LaTeX\
        without the use of any external programs. However, |pgfpages| can do
        quite a lot more than that. You can use it to put logos and watermark
        on pages, print up to 16 pages on one page, add borders to pages, and
        more.
\end{enumerate}


\subsection{如何阅读本使用手册}

本使用手册介绍了\tikzname的设计和使用。
章节组织是非常粗略地根据“用户友好性”进行的。
首先介绍了最简单和最常用的命令和子包,稍后介绍更底层和更深奥的特性。

如果你还没安装\tikzname,请先阅读安装部分。
其次,阅读教程部分应该是个好主意。
最后,你可能希望浏览下\tikzname的介绍。
通常,你不需要阅读基础层的部分。
如果你打算编写自己的前端,或者你希望将\pgfname移植到一个新的驱动程序,你只需要阅读系统层的部分。

系统提供的“public(公开)”命令和环境在全文中都有介绍。
在每个这样的介绍中,所描述的命令、环境或选项都以红色打印。
绿色的文本是可选的,可以忽略。


\subsection{作者和致谢}
\label{section-authors}

大部分的\pgfname系统及其文档是由Till Tantau编写的。
主要团队的另一名成员是Mark Wibrow,他负责例如\pgfname数学引擎、许多形状、装饰引擎和矩阵等。
第三个成员是Christian Feuers\"anger,他在手册中贡献了浮点库、图像外部化、扩展键处理和自动超链接等。

此外,Christophe Jorssen、Jin-Hwan Cho、Olivier Binda、Matthias Schulz、Ren\'ee Ahrens、Stephan Schuster和Thomas Neumann等也偶有贡献。

另外,许多人通过写电子邮件、发现缺陷或发送库和补丁为\pgfname系统做出了贡献。
非常感谢所有这些人,他们太多了,无法一一列举!


\subsection{寻求帮助}

当你需要关于\pgfname和\tikzname的帮助,请做如下工作:

\begin{enumerate}
    \item 阅读使用手册,至少是与你的问题有关的那部分。
    \item 如果这不能解决问题,试着看下\pgfname和\tikzname的Github开发页面(参加本文档的标题处)。
        也许有人已经报告了类似的问题,有人已经找到了解决方案。
    \item 在网站上你可以找到许多寻求帮助的论坛。
        在那里,你可以发言来帮助论坛、提出缺陷报告、加入邮件列表等等。
    \item 在你提交缺陷报告,尤其是关于安装的缺陷报告之前,请确认这确实是一个缺陷。
        特别是,看一下当你\TeX你的文件时生成的|.log|文件。
        这个|.log|文件应该显示所有正确的文件都是从正确的目录加载的。
        几乎所有的安装问题都可以通过查看|.log|文件来解决。
    \item \emph{作为最后一招}你可以试着给我(Till Tantau)发邮件,或者是Mark Wibrow,如果问题与数学引擎有关。
        我不介意收到邮件,我只是收到的邮件太多了。
        因此,我不能保证你的电子邮件将得到及时回复,甚至根本没有。
        如果你将邮件发送到\pgfname邮件列表,那么你的问题得到解决的几率会更高一些(当然,当我有时间时,我会阅读这个邮件列表并回答问题)。
\end{enumerate}
